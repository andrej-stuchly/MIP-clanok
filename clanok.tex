% Metódy inžinierskej práce

\documentclass[10pt,twoside,slovak,a4paper]{article}

\usepackage[slovak]{babel}
%\usepackage[T1]{fontenc}
\usepackage[IL2]{fontenc} % lepšia sadzba písmena Ľ než v T1
\usepackage[utf8]{inputenc}
\usepackage{graphicx}
\usepackage{url} % príkaz \url na formátovanie URL
\usepackage{hyperref} % odkazy v texte budú aktívne (pri niektorých triedach dokumentov spôsobuje posun textu)

\usepackage{cite}
\usepackage{times}

%\pagestyle{headings}

\title{Virtuálna realita vo vzdelávaní\thanks{Semestrálny projekt v predmete Metódy inžinierskej práce, ak. rok 2021/22, vedenie: Fedor Lehocki}} 

\author{Andrej Stuchlý\\[2pt]
	{\small Slovenská technická univerzita v Bratislave}\\
	{\small Fakulta informatiky a informačných technológií}\\
	{\small \texttt{xstuchly@stuba.sk}}
	}

\date{\small 4. november 2021} 



\begin{document}

\maketitle

\begin{abstract}
Technologický pokrok je v posledných rokoch veľmi rýchly a poskytol nám mnohé výdobytky novej doby, medzi inými aj virtuálnu realitu. Táto technologická novinka je nepochybne skvelá pre relax v podobe hier a filmov, avšak môže byť veľmi efektívna v profesionálnom živote v rámci štúdia a praxe v zamestnaní, obzvlášť pri práci z domu. Vo svojom článku chcem priblížiť základné informácie o virtuálnej realite a prepracovať sa k  praktickému využitiu vo výučbe v lekárstve, inžinierskom a sociálnom odvetví. Pri jednotlivých odvetviach štúdia by som rád rozobral praktické zručnosti a schopnosti nevyhnutné pre budúce zamestnanie, ktoré dokážu študenti nadobudnúť pomocou simulácii vo virtuálnej realite. Taktiež spomeniem jej dostupnosť a aj využitie ako ekonomicky výhodnú alternatívu finančne náročných projektov. 
\end{abstract}


\section{Úvod}
Virtuálna realita a jej všestranné využitie ma veľmi zaujalo a chcel som sa o ňom dozvedieť čo najviac, práve preto som sa rozhodol venovať práve tejto téme v mojom článku. V širokej spoločnosti sa o virtuálnej realite hovorí skôr ako o niečom, čo je predovšetkým určené na zábavu, či už vo forme hier alebo rôznych filmov a seriálov, preto by som sa chcel pozrieť na túto technológiu z praktickejšieho a užitočnejšieho uhla pohľadu - využitie na naučenie sa niečoho nového a prehĺbenie svojich vedomostí a schopností v profesijnom živote a vďaka tomu mať ešte lepší zážitok z výúčby nových praktických schopností.

Práve vzdelávaniu pomocou VR sa chcem v tomto článku do hĺbky venovať: v druhej časti \ref{druha} uvediem hlavné informácie o virtuálnej realite, jej možnostiach, dostupnosti a cene. Následne rozoberiem konkrétne využitie vo viacerých odvetviach vzdelávania, \ref{tretia} či už sa jedná o malé deti alebo vysokoškolských študentov z rôznych odvetví. Po tomto sa dostaneme do štvrtej kapitoly \ref{stvrta}, kde budem uvažovať o možnom využití VR pri externom štúdiu a taktiež dištančnej výúčbe, ktorá je, bohužiaľ, momentálne veľmi častá a bežná. Nakoniec všetky moje myšlienky o tejto téme a problematike zhrniem na záver. \ref{zaver}


\section{Virtuálna realita} \label{druha}

Na začiatok je dôležité uviesť, čo je vlastne virtuálna realita: najčastejšie sa uvádza,\cite{VR} že je to pokročilé ľudsko-počítačové rozhranie (human-computer interface), ktoré simuluje reálne prostredie. Používateľ sa môže voľne pohybovať v tomto prostredí, môže sa na veci pozerať z rôznych uhlov, siahnuť na ne, uchopiť ich a zmeniť ich. 

Môžeme povedať, že virtuálna realita je postavená na dvoch hlavných pilieroch: \cite{VR_kniha}:
\begin{itemize}
\item interaktivita - schopnosť narábať s prostredím a všetkým, čo poskytuje, všetko musí reagovať na činy účastníka, hlavná motivácia je poskytnutie človeku čo najväčšiu voľnosť v rámci daného virtuálneho prostredia
\item vnorenie - vhĺbenie do zážitku, odstránenie všetkých rušivých elementov a sústredenie sa na to, s čím treba pracovať; poskytnutie pútavého zážitku a zaujatie používateľa
\end{itemize}

Kedže hlavný cieľ je, aby sa účastník cítil, akoby bol naozaj na inom mieste je nevyhnutné, aby sa prepojili ľudské zmysly a svaly s týmto virtuálnym prostredím. Aby sa toto podarilo dosiahnúť s čo najväčšou autenticitou, používajú sa ako nástroje na ovládanie takéhoto prostredia okuliare pre virtuálnu realitu a dva ovládače, každý do jednej ruky. 
V dnešnej dobe je už táto technológia veľmi rozšírená a cena takéhoto setu je stále nižšia, čo má za následok vyššiu dostupnosť a stále viac ľudí, ktorí si ho zakúpia pre vlastné používanie. Toto je jeden z hlavných faktorov, ktoré môžu ovplyvniť, ako veľmi bude VR používané v bežnom živote a ako môže preniknúť do výučby ľudí a posunúť štúdium na ešte vyššiu úroveň.

\section{Vzdelávanie pomocou VR} \label{tretia}
Technológia virtuálnej reality dokáže byť veľmi prospešná pri vzdelávani a učení ľudí nových teoretických a hlavne praktických vedomostí a skúseností. Takisto je veľmi dobrá široká uplatniteľnosť tohto vynálezu - môže byť využitý ako prostriedok výučby pre malé deti, študentov vysokých škôl a taktiež aj zamestnancov firiem a pobočiek. 

Jedna z veľkých výhod je taktiež simulácia náročného projektu alebo experimentu  - vďaka tomuto vieme eliminovať stratu množstva času pri príprave pokusu, zháňaniu potrebných materiálov a nástrojov a taktiež aj risku nebezpečenstva, ktorý môže nastať pri skkutočných pokusoch: môžeme jednoducho simuláciu zrušiť a spustiť znovu. Jedna z veľkých výhod, ktorú je dôležité spomenúť, je taktiež aj skutočnosť, že vďaka VR sme schopní ušetriť množstvo finančných prostriedkov - môžme opakovať ten istý projekt viackrát bez potreby dodávky nových materiálov nevyhnutných pre skutočný experiment.

 V ďalších podkapitolách sa budem venovať konkrétnemu využitiu vrámci vyššie spomínaných vekových kategórii a takiež budem hovovriť o rôžnych odvetviach, ktoré dokážu výrazne profitovať zo simulácie reálneho prostredia.

\subsection{Výučba vysokoškolských študentov} \label{studenti}

\subsection{Kurzy pre zamestnancov} \label{zamestnanci}

\subsection{Pomoc s učením zdravotne znevýhodnených} \label{pomoc}

\section{VR a dištančná výuka} \label{stvrta}

Nie je vždy možné, aby bol človek schopný študovať prezenčne, v takom prípade často dochádza k externému štúdiu, prípadne dištančnej forme štúdia kvôli nepriaznivým okolnostiam (napríklad súčasná situácia na Slovensku, kedy je nevyhnutné študovať z domu kvôli pandémii Covidu). Vzhľadom k tomu, že človek nie je osobne prítomný na vyučovaní a nie je možné si viaceré praktické úlohy a experimenty vyskúšať, natíska sa otázka, či by tento problém dokázala vyriešiť virtuálna realita.

Čo sa týky externého štúdia, VR by mohla byť skvelý doplnok k štúdiu. Nakoľko takýto typ štúdia je od začiatku zameraný na dištančnú formu, bolo by možné zadovážiť si potrebnu technológiu na virtuálnu realitu ešte pred začiatkom štúdia a vďaka tomu by bolo vzedlávanie prepojené s priamym kontaktom a schopnosťou vyskúšať mnohé veci, ktoré by sa za normálnych okolností z pohodlia domova nedali vykonávať.

Avšak oproti tomu naša dištančná výuka by teraz nefungovala tak jednoducho - hlavný dôvod je, že nikto nebol na takúto situáciu pripravený a s najväčšou pravdepodobnosťou nemá doma technické vybavenie potrebné na takýto typ výučby. Ďalší problém je v otázke financii, nakoľko potrebné zariadenie na sfunkčnenie virtuálnej reality nie je lacná záležitosť a je možné, že viacerí študenti by neboli schopní alebo ochotní obetovať takú sumu peňazí za niečo, čo budú na vyučovanie potrebovať nevyhnutne iba počas dištančnej výuky. Napriek tomu si myslím, že so zvyšujúcou sa dostupnosťou a znižujúcou sa cenou VR setu je možné, že v budúcnosti sa z neho stane skoro samozrejmosť v domácnosti, tak ako teraz počítač alebo televízor. V takom prípade by sa aj v takejto náročnej situácii dala udržať veľmi vysoká kvalita štúdia.

\section{Záver} \label{zaver} 
Text\cite{Study}, A~\cite{Surgery} B \cite{Learning}C \cite{Engineering} D  \cite{Games} možno \emph{zdôrazniť kurzívou}.


%\acknowledgement{Ak niekomu chcete poďakovať\ldots}



\bibliography{literatura}
\bibliographystyle{unsrt}
\end{document}
