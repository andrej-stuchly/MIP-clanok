% Metódy inžinierskej práce

\documentclass[10pt,twoside,slovak,a4paper]{article}

\usepackage[slovak]{babel}
%\usepackage[T1]{fontenc}
\usepackage[IL2]{fontenc} % lepšia sadzba písmena Ľ než v T1
\usepackage[utf8]{inputenc}
\usepackage{graphicx}
\usepackage{url} % príkaz \url na formátovanie URL
\usepackage{hyperref} % odkazy v texte budú aktívne (pri niektorých triedach dokumentov spôsobuje posun textu)

\usepackage{cite}
\usepackage{times}

\pagestyle{headings}

\title{Virtuálna realita vo vzdelávaní\thanks{Semestrálny projekt v predmete Metódy inžinierskej práce, ak. rok 2021/22, vedenie: Fedor Lehocki}} 

\author{Andrej Stuchlý\\[2pt]
	{\small Slovenská technická univerzita v Bratislave}\\
	{\small Fakulta informatiky a informačných technológií}\\
	{\small \texttt{xstuchly@stuba.sk}}
	}

\date{\small 4. november 2021} 



\begin{document}

\maketitle

\begin{abstract}
Technologický pokrok je v posledných rokoch veľmi rýchly a poskytol nám mnohé výdobytky novej doby, medzi inými aj virtuálnu realitu. Táto technologická novinka je nepochybne skvelá pre relax v podobe hier a filmov, avšak môže byť veľmi efektívna v profesionálnom živote v rámci štúdia a praxe v zamestnaní, obzvlášť pri práci z domu. Vo svojom článku chcem priblížiť základné informácie o virtuálnej realite a prepracovať sa k  praktickému využitiu vo výučbe v lekárstve, inžinierskom a sociálnom odvetví. Pri jednotlivých odvetviach štúdia by som rád rozobral praktické zručnosti a schopnosti nevyhnutné pre budúce zamestnanie, ktoré dokážu študenti nadobudnúť pomocou simulácii vo virtuálnej realite. Taktiež spomeniem jej dostupnosť a aj využitie ako ekonomicky výhodnú alternatívu finančne náročných projektov. 
\end{abstract}



\section{Úvod}

Technológia v posledných dekádach veľmi napreduje a poskytuje nám mnoho skvelých vynálezov, medzi ktoré môžeme nepochybne zaradiť aj virtuálnu realitu, alebo skrátene VR - prostriedok skvelý ako na zábavu, tak aj na osobný rozvoj jedinca a vzdelávanie. Práve druhému spomenutému sa chcem v tomto článku do hĺbky venovať: v druhej časti \ref{druha} uvediem hlavné informácie o virtuálnej realite, jej možnostiach, dostupnosti a cene. Následne sa dostanem do tretej časti\ref{tretia}, kde rozoberiem konrkrétne využitie vo viacerých odvetviach vzdelávania, či už sa jedná o malé deti alebo vysokoškolských študentov z rôznych odvetví. Po tomto sa dostaneme do štvrtej časti \ref{stvrta}, kde budem uvažovať o možnom využití VR pri externom štúdiu a taktiež dištančnej výúčbe, ktorá je, bohužiaľ, momentálne veľmi častá a bežná. Nakoniec všetky moje myšlienky o tejto téme a problematike zhrniem na záver \ref{zaver}.



\section{Virtuálna realita} \label{druha}

Na začiatok je dôležité uviesť, čo je vlastne virtuálna realita: je to pokročilé human-computer interface, ktoré simuluje reálne prostredie.\cite{VR} Používateľ sa môže voľne pohybovať v tomto prostredí, môže sa na veci pozerať z rôznych uhlov, siahnuť na ne, uchopiť ich a zmeniť ich. 

Medzi hlavné základy virtuálnej reality patria:
\begin{itemize}
\item interaktivita - schopnosť narábať s prostredím a všetkým, čo poskytuje
\item vnorenie - vhĺbenie do zážitku, odstránenie všetkých rušivých elementov a sústredenie sa na to, s čím treba pracovať
\end{itemize}

\section{Vzdelávanie pomocou VR} \label{tretia}


\section{VR a dištančná výuka} \label{stvrta}

Nie je vždy možné, aby bol človek schopný študovať prezenčne, v takom prípade často dochádza k externému štúdiu, prípadne dištančnej forme štúdia kvôli nepriaznivým okolnostiam (napríklad súčasná situácia na Slovensku, kedy je nevyhnutné študovať z domu kvôli pandémii Covidu). Vzhľadom k tomu, že človek nie je osobne prítomný na vyučovaní a nie je možné si viaceré praktické úlohy a experimenty vyskúšať, natíska sa otázka, či by tento problém dokázala vyriešiť virtuálna realita.

Čo sa týky externého štúdia, VR by mohla byť skvelý doplnok k štúdiu.

Avšak oproti tomu naša dištančná výuka by pravdepodobne nefungovala tak jednoducho - hlavný dôvod je, že nikto nebol na takúto situáciu pripravený a s najväčšou pravdepodobnosťou nemá doma technické vybavenie potrebné na takýto typ výučby.Ďalší problém je v otázke financii, nakoľko potrebné zariadenie na sfunkčnenie virtuálnej reality nie je lacná záležitosť a je možné, že viacerí študenti by neboli schopní alebo ochotní obetovať takú sumu peňazí za niečo, čo budú na vyučovanie potrebovať nevyhnutne iba počas dištančnej výuky.


\section{Záver} \label{zaver} 
Text\cite{Study}, A~\cite{Surgery} B \cite{Learning}C \cite{Engineering} D  \cite{Games} možno \emph{zdôrazniť kurzívou}.


%\acknowledgement{Ak niekomu chcete poďakovať\ldots}



\bibliography{literatura}
\bibliographystyle{unsrt}
\end{document}
